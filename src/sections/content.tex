%! Author = jasmincapka
%! Date = 14.07.22

Imagine the following situation~\parencite[][1-2]{patreon}:
You are trying to gather public data to conduct an analysis on, but have to rely on web-scraping because the data is not directly available.
Because of this process, you cannot be sure about the completeness of the data and the validity of the resulting conclusions.
Then, because of a malicious hack, data including the needed public data for the analysis but also private data suddenly becomes directly available.

The question now arises, whether it is appropriate to use data from illicit sources for the planned research.
Therefore, this discussion focuses on the ethics of using hacked data in data science.\\
Hacked data in this context means ``data obtained in an unauthorized manner through illicit access to a computer or computer network''~\parencite[][744]{nature} and, thus, the exploitation of a vulnerability.
Additionally, this paper assumes that the person potentially using the hacked data is in no way responsible for the extraction of the hacked data.

\section*{Reconstruction of problem}

This section focuses on the background, why using hacked data in data science needs ethical considerations according to~\cite[][745]{nature}:

The ``legal and ethical boundaries of using hacked data for research are blurred''~\parencite[][745]{nature}.
As hacked data is available online, it is in the public domain which makes it per se permissible to use as the usage is not against the law.\\
Still, only because it is legally permissible, it does not account for the social responsibility of scientists and their moral obligations.
It has to be determined, whether the use of hacked data is compatible with research ethics.\\
Currently, there is few ``ethical guidance on the use of hacked data for research''~\parencite[][745]{nature}.
There are Institutional Review Boards or similar bodies such as Research Ethics Committees to ensure ethical research regarding personal data and prevent scientific misconduct.
Still, their oversight function can be bypassed.\\
There has been some effort to define best practices among researches, but the usage of hacked data has not yet been addressed directly.
This means, no general normative guidance is provided.

Therefore, the question need to be addressed, whether, and under which conditions, research using hacked data might be morally justifiable~\parencite[][745]{nature}.

\section*{Reconstruction of arguments}

The three most relevant papers that bring arguments in favor of and against the usage of hacked data in data science are:
\cite{patreon} examine the topic from their own perspective regarding the previously mentioned case.
\cite{nature} cover the general debate and develop a set of guidelines for the potential usage of hacked data.
\cite{acm} weights arguments taking the SCM Code of Ethics into account.

The first argument in favor of the usage is that the data is already public~\parencite[][5]{patreon}.
Therefore, it can be used analogously to any other publicly accessible data~\parencite[][23]{acm}.
The illegality of its release would then be only relevant for evaluating the data quality.\\
Another argument is ``cross-domain consistency''~\parencite[][746]{nature}.
As hacked data is used in other professions like journalism, it should be usable in data science as well~\parencite[][745]{nature}.\\
Additionally, the data is of public value, meaning, an analysis can serve the public good and compensate for harms of a data breach~\parencites[][5]{patreon}[][23]{acm}.
In this case, the potential public benefits from using hacked data outweigh the harms caused by retrieving it~\parencite[][23]{acm}.\\
Furthermore, the data has a unique character as it may not be retrievable via any other conventional method in terms of accuracy and amount of data~\parencites[][745]{nature}[][5]{patreon}.\\
Lastly, even if the data could be attained otherwise, using hacked data is an ``optimization of resources''~\parencite[][746]{nature}.
Otherwise doing the same research would cost a large amount of resources for conventionally extracting data that is already publicly available~\parencite[][745]{nature}.

In contrast to those arguments, using hacked data in data science might cause secondary harm~\parencites[][746]{nature}[23]{acm}.
The condition of hacked data being publicly available does not eliminate the potential of additional risks through further analysis.
Research might still cause further harm to people, either individually or on a group level.\\
Furthermore, using hacked data is a violation of the expectation of privacy of the people, the data is about~\parencite[][5]{patreon}.
There is only limited ability to differentiate between public and private, personally identifiable information within hacked data.
This makes it impossible to respect data privacy although the data is de-identified before publication~\parencite[][746]{nature}.\\
In addition, ``scientists [are] not allowed to publish information without explicit and informed consent''~\parencite[][745]{nature}.
As consent is missing in case of hacked data, using hacked data would be using people's data without consent~\parencite[][5]{patreon}.\\
Another argument is that hacked data might be biased or of lower quality than legitimately obtained data~\parencite[][746]{nature}.
This might lead to a lowering of scientific standards regarding integrity and research ethics and have a negative effect on the perception of science by society.
Unethical methods used to attain hacked data might mark any research using it as unethical~\parencite[][24]{acm}.\\
Additionally, the using hacked data is not mandatory~\parencite[][5]{patreon}.
Other data can be ethically extracted and used, even if that means not analyzing a certain topic at all or researching from a different perspective.\\
Lastly, the usage of hacked data might legitimize criminal activity as ``research''~\parencites[][5]{patreon}[][24]{acm}.
Therefore, refusing to use hacked data is an important statement about conducting research ethically and sets an example for other data scientists in the future.

\section*{Possible framework}


research using hacked data should
be deemed unethical, hence impermissible, by default as the direct
and indirect negative consequences of condoning the use of such
data in science seem to outweigh the benefits.~\parencite[][747]{nature}

- Uniqueness~\parencite[][747]{nature}
  - prove that no alternative data collection strategy was viable for their research question
- Risk-benefit assessment~\parencite[][747]{nature}
  - high social value, benefits clearly outweigh the harms
  - processing the hacked data involves no more-than-minimal risk to the data subjects, and
  - probability and magnitude of harm or discomfort anticipated in the research are no greater than that normally encountered in the data generators’ daily (online) life
- consent~\parencite[][747]{nature}
  - researchers must obtain explicit and informed consent from data generators before using the data for research
  - Processing pseudonymized or anonymized data for scientific purposes may be permissible
- Traceability~\parencite[][747]{nature}
  - must provide a record of how and where all data have been obtained.
- Privacy~\parencite[][747]{nature}
  - do they have accessed personally identifiable data without consent
  - what safeguards sre in place to prevent reidentification
  - how will individuals be informed about insecure data transmission
  - who is legally responsible
  - not make repository public
- IRB approval~\parencite[][748]{nature}
  - not self assessment
  - Independent oversight bodies should always determine whether an exception to the unethical-by-default rule should be admitted
  - ethics review should be mandatory for all
  - journals should require disclosure statement


\section*{Eigene Meinung/critical thinking zu solution}

very restrictive approach
- first three are prerequisites
- last two are information

not possible to stop it
- better implement rule not to publish it -> journals do not have to follow

do ethical committees have that much time?

falls back on:
Code of ethics not established
- not only research ethics, but data science ethics
- data science as own profession not fully included in research
- Aufnehmen in potential Code of Ethics
  - not only possible Vorschläge
- Aufmerksamkeit schaffen


Rückschluss auf Intro Paper
