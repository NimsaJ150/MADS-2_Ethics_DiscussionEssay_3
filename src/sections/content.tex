%! Author = jasmincapka
%! Date = 31.03.22

This discussion focuses on the argument of~\cite{vredenburgh} that the discussion of algorithmic bias should not focus on fairness, but on justice.

Algorithmic bias: "deviation of an
algorithm’s performance from what is required by some standard"~\parencite[][1]{vredenburgh}
moral sense
"the encoding of wrongfully discriminatory social patterns into an
algorithm"~\parencite[][2]{vredenburgh}

\section*{Explain, in your own words and using your own examples, what she means by “fairness” and “justice”}

Justice
- moral norms that govern our most fundamental institution
- "moral norms that govern our basic societal institutions, including norms of distribution and norms of respect"~\parencite[][3]{vredenburgh}
- "term that encapsulates a number of moral commitments about how society’s key institutions ought to treat people"~\parencite[][11]{vredenburgh}
- "moral standards that ought to structure major societal institutions, such as legal, political, and economic institutions"~\parencite[][11]{vredenburgh}
- "standards determine people’s obligations, entitlements, opportunities, and burdens"~\parencite[][12]{vredenburgh}
- "shape people’s life trajectories, as well as their ideas about justice"~\parencite[][12]{vredenburgh}
- Example!!!


"Fairness is one of the moral values that ought to shape our basic societal institution, but it is not the only value of justice"~\parencite[][3]{vredenburgh}
"one type of standard of justice"~\parencite[][11]{vredenburgh}


Fairness
- equal consideration of claims
- "whether alike individuals are treated equally"~\parencite[][3]{vredenburgh}
- "respecting equal claims, as well as respecting claims in proportion to their strength"~\parencite[][13]{vredenburgh}
- "Reasons of fairness are grounded in the moral equality of people."~\parencite[][13]{vredenburgh}
- "Thus, fairness is a matter of proportional equality, or giving people their due relative to what is owed to them."~\parencite[][13]{vredenburgh}
- "A fair outcome is a distribution of resources, material or not, that respects people’s claims"~\parencite[][13]{vredenburgh}
- Example!!!

"Fairness is a central moral concept in our thinking about justice and AI"~\parencite[][13]{vredenburgh}
"However, it is not the only moral concern one might have"~\parencite[][13]{vredenburgh}

\section*{reconstruct her arguments for her thesis}

Thesis: The important dimension for moral evaluation of algorithms is justice, not fairness.
Fairness w/o justice has no value

\section*{focus on the policy recommendations she makes for just AI}



"Take a values-first approach to bias interventions"~\parencite[][20]{vredenburgh}

"De-couple decision processes"~\parencite[][22]{vredenburgh}

"Model structural injustice"~\parencite[][23]{vredenburgh}

"Better data"~\parencite[][24]{vredenburgh}

"Replace decision thresholds with more (weighted) lotteries"~\parencite[][24]{vredenburgh}



\section*{Which of these recommendations do you find the most important and why}

\section*{Illustrate what that recommendation would mean in practice by using your own example.}

