%! Author = jasmincapka
%! Date = 14.07.22

\section*{General debate}
Imagine the following situation~\parencite[][1-2]{patreon}:
You are trying to gather public data to conduct an analysis on, but have to rely on web-scraping because the data is not directly available.
Because of this process, you cannot be sure about the completeness of the data and the validity of the resulting conclusions.
Then, because of a malicious hack, data including the needed public data for the analysis but also private data suddenly becomes directly available.

The question now arises, whether it is appropriate to use data from illicit sources for the planned research.
Therefore, this discussion focuses on the ethics of using hacked data in data science.

Hacked data in this context means "data obtained in an unauthorized manner through illicit access to a computer or computer network"~\parencite[][744]{nature} and, thus, the exploitation of a vulnerability.
Additionally, this paper assumes that the person potentially using the hacked data is in no way responsible for the extraction of the hacked data.

\section*{Reconstruction of problem}

Using hacked data in data science needs ethical considerations, because the "legal and ethical boundaries of using hacked data for research are blurred"~\parencite[][745]{nature}.
As hacked data is available online, it is in the public domain which makes it per se permissible to use as the usage is not against the law.

research integrity is not merely reducible to lawfulness.~\parencite[][745]{nature}
- use of data compatible with research ethics?
- Scientists have a social responsibility and
a moral obligation to consider the ethics of their work beyond mere
compliance with current regulations

Current ethical guidance on the use of hacked data for research
is scarce.~\parencite[][745]{nature}
- Institutional Review Boards (IRBs) or analogous bodies such as Research Ethics
Committees
- competent authorities designated to ensure
the ethical oversight of research (scientific misconduct and or unethical research)
- oversight function can be bypassed

academic societies and professional associations
have made an effort to promote best practices~\parencite[][745]{nature}
- none of these guidance documents directly addresses the hacked data dilemma
- no general normative guidance provided

Therefore, the question arises, whether, and under which conditions, research using hacked data might be morally justifiable~\parencite[][745]{nature}.

\section*{Reconstruction of arguments}

Three paper:
- focus on own case regarding public data
- focus on hacked data and
- general case

Pro:
- Data is public, like a newspaper~\parencite[][5]{patreon}
  - since the data is already publicly accessible, it can be used the same way as any other publicly accessible data.~\parencite[][23]{acm}
  - methods and motives are only relevant for evaluating its quality~\parencite[][23]{acm}
- public value
  - We hope to serve the public good via our work~\parencite[][5]{patreon}
  - compensate for harms caused by data breach~\parencite[][745]{nature}
  - potential benefits to society from utilizing that data outweigh the harms caused by obtaining it~\parencite[][23]{acm}
- Uniqueness of data~\parencite[][745]{nature}
  - This is the data we want, but we can’t get it via other methods~\parencite[][5]{patreon}
  - would not be otherwise attainable using conventional methods.~\parencite[][745]{nature}
- cross-domain consistency~\parencite[][746]{nature}
  - hacked data are widely used in other professions such as journalism.~\parencite[][745]{nature}
- optimization of resources~\parencite[][746]{nature}
  - using hacked data in research seems to solve an optimization problem captured by the motto ‘once the data are there, we better use them~\parencite[][745]{nature}
  - doing research that would cost a large amount of resources otherwise~\parencite[][745]{nature}

Con:
- Secondary harm~\parencite[][746]{nature}
  - scientific analysis might cause further harm to data subjects, either individually or on a group level.~\parencite[][746]{nature}
  - fact the data from a security breach is publicly accessible does not mean using it in research does not create additional risks to those it describes~\parencite[][23]{acm}
- Privacy breach~\parencite[][746]{nature}
  - Researchers have a limited capability to distinguish between public and private information within the hacked data.~\parencite[][5]{patreon}
  - Violating users’ expectation of privacy.~\parencite[][5]{patreon}
  - May see private data when cleaning the data.~\parencite[][5]{patreon}
  - although researchers de-identify data before publication, they may have access to personally identifiable information during their analysis owing to identifiers contained in the dataset~\parencite[][746]{nature}
- Lack of consent~\parencite[][746]{nature}
  - Using people’s data without consent.~\parencite[][5]{patreon}
  - scientists not allowed to publish information without explicit and informed consent~\parencite[][745]{nature}
- Lowering quality standards~\parencite[][746]{nature}
  - hacked data may be subject to greater risks of bias and lower quality than legitimately obtained data~\parencite[][746]{nature}
  - Lowering the standards of scientific integrity and research ethics is not simply morally problematic in itself, but may also have a negative effect on the role of science within society~\parencite[][746]{nature}
  - unethical methods used to obtain the data ‘taints’ both the data itself and any research using it as immoral~\parencite[][24]{acm}
- We want this data, but we don’t need it. Other data can be ethically collected and used.~\parencite[][5]{patreon}
  - not mandatory to use this data
- Perhaps legitimizing criminal activity.~\parencite[][5]{patreon}
  - grants the methods and those who used them an unacceptable legitimacy as ‘researchers’~\parencite[][24]{acm}
  - refusing to use such data is an important statement about conducting research ethically and deters researchers from using such methods to obtain data in the future.~\parencite[][24]{acm}

\section*{Possible framework}

research using hacked data should
be deemed unethical, hence impermissible, by default as the direct
and indirect negative consequences of condoning the use of such
data in science seem to outweigh the benefits.~\parencite[][747]{nature}

- Uniqueness~\parencite[][747]{nature}
  - prove that no alternative data collection strategy was viable for their research question
- Risk-benefit assessment~\parencite[][747]{nature}
  - high social value, benefits clearly outweigh the harms
  - processing the hacked data involves no more-than-minimal risk to the data subjects, and
  - probability and magnitude of harm or discomfort anticipated in the research are no greater than that normally encountered in the data generators’ daily (online) life
- consent~\parencite[][747]{nature}
  - researchers must obtain explicit and informed consent from data generators before using the data for research
  - Processing pseudonymized or anonymized data for scientific purposes may be permissible
- Traceability~\parencite[][747]{nature}
  - must provide a record of how and where all data have been obtained.
- Privacy~\parencite[][747]{nature}
  - do they have accessed personally identifiable data without consent
  - what safeguards sre in place to prevent reidentification
  - how will individuals be informed about insecure data transmission
  - who is legally responsible
  - not make repository public
- IRB approval~\parencite[][748]{nature}
  - not self assessment
  - Independent oversight bodies should always determine whether an exception to the unethical-by-default rule should be admitted
  - ethics review should be mandatory for all
  - journals should require disclosure statement


\section*{Eigene Meinung/critical thinking zu solution}

very restrictive approach
- first three are prerequisites
- last two are information

not possible to stop it
- better implement rule not to publish it -> journals do not have to follow

do ethical committees have that much time?

falls back on:
Code of ethics not established
- not only research ethics, but data science ethics
- data science as own profession not fully included in research
- Aufnehmen in potential Code of Ethics
  - not only possible Vorschläge
- Aufmerksamkeit schaffen


Rückschluss auf Intro Paper
